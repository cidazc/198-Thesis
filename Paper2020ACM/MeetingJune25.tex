Marami to hahahaha get ready mostly sa proof of correctness

Whole Paper:
✓ 1) Make it 2 column format
✓ 2) Add sir El King Morado - UP Philippine Genome Center
✓ 3) Remove references to page numbers
✓ 4) Put figures on top of a column (see what sir el king sent)

✓ Abstract
✓ - okay content
✓ - fix typo Data nods

✓Intro Pg 1
✓ - Remove sentence: [the goal of this intro]
✓- Genomics study of blank. To generate data you need sequencing. But the data is big & needs effort. To sequence a lot, tech evolved and NGS gave high throughput
✓- Intro should be 2 columns lang. Make each paragraph a few sentences instead.
✓- Intro goal: layoutht background of the work
✓- Flow: Idea of A lot of data -> Problems -> Proposed Solution
✓- remove subsections in inntro (do this last)


✓Intro Pg 2
✓ - 3,000,000,000 -> 3 billion
✓- Shorten HGP section. Combine HGP + NGS to one paragraph.
✓ - Fix typo: Genomics is is

✓-Intro: Database
✓- NGS -> high voluminous data -> need to store it in a digital format -> retrieve it for analysis


✓ Overall info guide: 
✓ - tama na yung flow
✓ - lessen info
✓ - compress database to 1 or 2 sentences. Then mention centralized/distributed database.

✓ Into: CAP
✓ - shorten and make it 1 paragraph
✓ - only describe the most important principle and why
✓- CID: Carlll paread through nalang ng nagawa ko

Application Architecture: (Move to Theoretical Framework)
✓ - Discuss Client Server in 1 sentence
✓- Discuss P2P in 1 sentence
✓- Describe Hybrid 
✓- First we need to discuss 2 basic architecture: CS, P2P. 
✓- CS is blank. 
✓- P2P is blank. 
✓- Our proposed system will be hybrid getting these aspects from CS and P2P

RRL:
- make the first paragraph say: We review blank. We would like to focus on blank and blank. In the end we would like to make a comparitive analysis. 
- Move Dat (move to theoretical framework)
(Todo monday) - Table Comparing the diff Databass, Describe each column narratively, and describe some conclusions from the table
- Answer the question: anong halaga ng columns? 
- Discuss each column CS/P2P/H, and the data on each column
✓ - FASTA rename to Biological

Review of Related Literature:
- the review tells you you are meant to critique the work you have seen
- here summarize gaps 
✓ - copy the problems to here


80 percent ✓ Problem:
✓ - move the problems bullet points to RRL. Replace problem bullet points with a paragraph summary 
✓ - summarize the gaps. To review the current implementation has problems in size.
- To summarize, centralized databases suffer from these problems. Therefore the research aims to focus on developing a database with these features
✓ - Problem Statement: Central thesis, one unifying question. May certain level of measurability.

Objectives:
- specify metrics, activities, na once you do this succeed.
✓ - the bullet points "Resolve problems blank of blank" move them to a paragraph below objectives
✓ - For the paragraph below: First objective addresses these problems, second objective addresses these limitations blank.
✓- keep first 2 objectives, the last is: to evaluate the theoretical properties of the system based on the correctness and its speed
✓ - framework and APIs move to theoretical framework. 


✓Scope
✓- 2 things: 
✓- 1) create a theoretical design for a proposed system
✓- 2) evaluate the theoretical running time and speed of the operations.
✓- define here what you mean by proposed system and base system (
✓- base system means FTP retrieval from NCBI

✓ Limitation
✓- make limitations section
✓- no implement
✓- be transparent, time constraints and pandemic
✓- no security and authentication


Theoretical Framework
✓ - should have a very basic flow
✓ - discusses abstractly anong paglalaruan na ideas
✓ - Discuss Hybrid,CS, P2P in short sentences
✓ - Network, forego layers 1-3. Briefly discuss layers 4 and 5 and their importance.
✓ - Discuss how retrieval operations rely on layers 4 and 5 (FTP for NCBI, torrent??)
✓ - good flow, but too many info, need to shorten it.
✓ - (half check, since this still needs to be moved to theoretical framework) say we will be developing an information system. And all information systems have GET,POST,PUT,DELETE functionalities.


Summary of the System
✓ - call this section caDDS
- start with an outline for the entire section (This section will describe the different operations in ...)
- say: The retrieval mechanism is inspired by SeqTorr. This section describes the proposed system. This is a hybrid focused or distributed system

✓ Community
✓ - this section is good
✓- move to 7.2 or after

System Description
- for each photo (GET, POST...) , need one paragraph explaining each one. 1 paragraph per operation (GET, POST...)
- for each photo outline what happens in 1) 2) 3)


Proof of Correctness (BIGTIME CHANGE + STUFF TO LEARN)
- how do you guarantee that a get does what it is intended to do:
- cite preconditions, post conditions,all the paths from pre conditions to post conditions and how it would always be true
- like an if else
- shows the path from pre condition to post condition
- set pre conditions, assumptions, then show how it becomes postconditions (state after)


Sample:
Assume do not factor in physical state of the network, it is error free
for GET
input of this operation: search key
output of this operation: data transferred to the user

Master node during search has 2 possible output
1) at least 1 hit
2) not in the database

After having at least 1 hit
-> sends the link to user
-> user clicks to link if he wants
-> user presses link
-> user makes a download request to master
-> master redirects user to data node
-> data node sends data to user

Precondition: User chooses to download
Postcondition: User retrieves files

User inputs:search key
master outputs: results report

How to know master will send the correct file?
1) each metadata has a unique identifier
2) guarantees exact file since each file has a unique identifier

How to know it will reach the correct user
1) Each user has a unique identifier (IP Address)

Theoretical Analysis
- Mention base system means FTP from NCBI (will be stated in scope but good to state again here) ...
- Mention why only get/ retrieval is used. Since all operations can be broken up into small retrieval operations
- abstract size of a payload
- cite that network is assumed to be error free
- articulate assumptions better
- move the final comparisons here

Conclusion Template:
✓ - make a conclusion section
- say what you did
- summarize analysis of results
- 2 to 3 sentences of what you did/ problem
- 2 sentences correctness and speed
- correctness: system operations
- speed: faster/slower
- shown to do what is expected based on preconditions

✓ Reccomendations
- okay

✓ Acknowledgements -> Remove from Appendix

✓ Biblio
✓ - Make it APA format

Presentation - after naaa
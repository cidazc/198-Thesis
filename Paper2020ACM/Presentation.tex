%%%%%%%%%%%%%%%%%%%%%%%%%%%%%%%%%%%%%%%%%
% Beamer Presentation
% LaTeX Template
% Version 1.0 (10/11/12)
%
% This template has been downloaded from:
% http://www.LaTeXTemplates.com
%
% License:
% CC BY-NC-SA 3.0 (http://creativecommons.org/licenses/by-nc-sa/3.0/)
%
%%%%%%%%%%%%%%%%%%%%%%%%%%%%%%%%%%%%%%%%%

%----------------------------------------------------------------------------------------
%	PACKAGES AND THEMEShttps://www.overleaf.com/project/5dbfadb88b2560000102ab44
%----------------------------------------------------------------------------------------

\documentclass{beamer}

\mode<presentation> {

% The Beamer class comes with a number of default slide themes
% which change the colors and layouts of slides. Below this is a list
% of all the themes, uncomment each in turn to see what they look like.

%\usetheme{default}
\usetheme{AnnArbor}
%\usetheme{Antibes}
%\usetheme{Bergen}
%\usetheme{Berkeley}
%\usetheme{Berlin}
%\usetheme{CambridgeUS}
%\usetheme{Copenhagen}
%\usetheme{Darmstadt}
%\usetheme{Dresden}
%\usetheme{Frankfurt}
%\usetheme{Goettingen}
%\usetheme{Hannover}
%\usetheme{Ilmenau}
%\usetheme{JuanLesPins}
%\usetheme{Luebeck}
%\usetheme{Madrid}
%\usetheme{Malmoe}
%\usetheme{Marburg}
%\usetheme{Montpellier}
%\usetheme{PaloAlto}
%\usetheme{Pittsburgh}
%\usetheme{Rochester}
%\usetheme{Singapore}
%\usetheme{Szeged}
%\usetheme{Warsaw}

% As well as themes, the Beamer class has a number of color themes
% for any slide theme. Uncomment each of these in turn to see how it
% changes the colors of your current slide theme.

%\usecolortheme{albatross}
%\usecolortheme{beaver}
%\usecolortheme{beetle}
%\usecolortheme{crane}
%\usecolortheme{dolphin}
%\usecolortheme{dove}
%\usecolortheme{fly}
%\usecolortheme{lily}
%\usecolortheme{orchid}
%\usecolortheme{rose}
%\usecolortheme{seagull}
%\usecolortheme{seahorse}
%\usecolortheme{whale}


%\usetheme{Boadilla}
\usecolortheme{wolverine}

%\setbeamertemplate{footline} % To remove the footer line in all slides uncomment this line
\setbeamertemplate{footline}[page number] % To replace the footer line in all slides with a simple slide count uncomment this line

\setbeamertemplate{navigation symbols}{} % To remove the navigation symbols from the bottom of all slides uncomment this line
}

\setbeamertemplate{bibliography item}{\insertbiblabel}


\usepackage{graphicx} % Allows including images
\usepackage{booktabs} % Allows the use of \toprule, \midrule and \bottomrule in tables
%\usepackage {tikz}
\usepackage{tkz-graph}
\usepackage{bibentry}


\usepackage[
backend=biber,
style=numeric,
sorting=ynt
]{biblatex}
 
\documentclass[xcolor=table]{beamer} 

\bibliography{bibfile}

\graphicspath{ {images/} }

\GraphInit[vstyle = Shade]
\tikzset{
  LabelStyle/.style = { rectangle, rounded corners, draw,
                        minimum width = 2em, fill = yellow!50,
                        text = red, font = \bfseries },
  VertexStyle/.append style = { inner sep=5pt,
                                font = \normalsize\bfseries},
  EdgeStyle/.append style = {->, bend left} }
\usetikzlibrary {positioning}
%\usepackage {xcolor}
\definecolor {processblue}{cmyk}{0.96,0,0,0}

% show sections only
\setcounter{tocdepth}{1}
% show subsections
%\setcounter{tocdepth}{2}
% show subsubsections
%\setcounter{tocdepth}{3}
\AtBeginSection[]
{
    \begin{frame}
        \frametitle{Table of Contents}
        \tableofcontents[currentsection]
    \end{frame}
}

%\AtBeginSubsection[]
%{
%    \begin{frame}
%        \frametitle{Table of Contents}
%        \tableofcontents[currentsection,currentsubsection]
%    \end{frame}
%}

%----------------------------------------------------------------------------------------
%	TITLE PAGE
%----------------------------------------------------------------------------------------

\title[caDDS]{caDDS: Community Assisted Distributed Database for Sequences} 

%%%%%%%%%%%%%%%%%%%%%%%%%%% INFO %%%%%%%%%%%%%%%%%%%%%%%%%%%%%%%%%%%%%%%%%%%%%%%%%

\author{Azcarraga, Araya} % Your name
\institute[Department of Computer Science, University of the Philippine - Diliman] % Your institution as it will appear on the bottom of every slide, may be shorthand to save space
{
University of the Philippine - Diliman\\ % Your institution for the title page
\medskip
}
\date{\today} % Date, can be changed to a custom date


%----------------------------------------------------------------------------------------
%	PRESENTATION SLIDES
%----------------------------------------------------------------------------------------


\begin{document}

%%%%%%%%%%%%%%%%%%%%%%%%%%%%% TITLE %%%%%%%%%%%%%%%%%%%%%%%%%%%%%%%%%%%%%%%%%%%%%%
\begin{frame}
\titlepage % Print the title page as the first slide
\end{frame}


%%%%%%%%%%%%%%%%%%%%%%%%%%%%% OVERVIEW %%%%%%%%%%%%%%%%%%%%%%%%%%%%%%%%%%%%%%%%%%%%%%
\begin{frame}
\frametitle{Overview} % Table of contents slide, comment this block out to remove it
\tableofcontents % Throughout your presentation, if you choose to use \section{} and \subsection{} commands, these will automatically be printed on this slide as an overview of your presentation
\end{frame}

%%%%%%%%%%%%%%%%%%%%%%%%%%%%% INTRO + RRL %%%%%%%%%%%%%%%%%%%%%%%%%%%%%%%%%%%%%%%%%%%%%%
\begin{frame}{Abstract}
  Genomic data the past few years have grown exponentially with the technology barely keeping up. The problem faced by genome researchers is the large data set and difficulty transferring the files. Previous distributed databases are either not meant for genomic data, or difficult to replicate. This paper lays the groundwork for a distributed database that is designed to easily scale to accommodate bigger storage, and also reduces the data speed bottleneck faced by centralized databases. The proposed system has a master node and data nodes. Data nodes should theoretically increase data transfer speeds even as the number of users increase. (to be updated based on conclusion) While current databases(e.g. NCBI) have more data, more users, and a live platform. The proposed system is aimed towards a smaller community, with more frequent and localized data, which the data storage can easily be expanded as the need arises. A recommendation is to implement the system and make the code public for the use of other communities.
\end{frame}

\begin{frame}{Genes}
\end{frame}

\begin{frame}{Genomics}
\end{frame}

\begin{frame}{Sequencing}
\end{frame}

\begin{frame}{Next Generation Sequencing}
\end{frame}

\begin{frame}{Growing problem of storage}
\end{frame}

\begin{frame}{Database}
\end{frame}

\begin{frame}{Database Definition}
\end{frame}

\begin{frame}{Database Principles}
\end{frame}

\begin{frame}{Database Types}
\end{frame}

\begin{frame}{Application Architecture}
\end{frame}

\begin{frame}{SeqTorr}
\end{frame}

\begin{frame}{BioTorrents}
\end{frame}

\begin{frame}{PeerDB}
\end{frame}

\begin{frame}{base system: NCBI}
\end{frame}

\begin{frame}{Problem}
\end{frame}

\begin{frame}{Objectives}
\end{frame}

\begin{frame}{Scope}
\end{frame}

\begin{frame}{System}
\end{frame}

\begin{frame}{Network}
\end{frame}

\begin{frame}{Summary}
\end{frame}

\begin{frame}{Community}
\end{frame}

\begin{frame}{Speed Comparisons}
\end{frame}


\begin{frame}{Comparisons}
\end{frame}


\begin{frame}{Recommendations}
\end{frame}

\begin{frame}
\Huge{\centerline{The End}}
\small{\centerline{Thank you for coming}}
\end{frame}

\section{Bibliography}

\begin{frame}[allowframebreaks]
\printbibliography[heading=none]
\end{frame}


\end{document}


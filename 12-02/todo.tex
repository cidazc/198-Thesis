\documentclass{article}

\begin{document}
\title{Week Nov 25-30}

\maketitle
\section{To Do?}

\begin{itemize}
    \item Paper
    \item Presentation
\end{itemize}

TITLE??


\section{Flow}

DONT PUT ANYTHING DI MO KAYA IEXPLAIN \\
\section{BASIC OUTLINE}
Intro \\
1) DNA unwinds to a long string \\
2) normal intro \\
4) Transition for eed for a database \\
5) 3 types of Database\\
5a) Normal (discuss din the RRL found \\
5b) Distributed \\
5c) P2P \\

Problem \\
6) Overview of all the problems found \\

7) synthesize into a unified problem statement \\

Objectives \\
8) connect problem statement to objectives 9) \\ SAY TITLE


Theoretical Fr. \\
9) This is how we want to attack \\
10) Discuss network stuff, peer2peer \\


Metho. \\

\section{IN DEPTH OUTLINE}
1st Slide - Picture how DNA can wrap around the moon
...
Sequencing Slide \\
    % transition to determine sequence of letters
    % highlight the output of sequencing runs, how many reads
    % volume can reach 100 gig for a single run or dataset
    % ^there is a need to have an organized way of storing things
    
Next Generation Sequencing Slide \\
    % speed of generating data, di at par yung speed of analysis
    % given computational computations,cant analyze fast
    % highlights need for a database

Seq Data vs Time graph Slide \\

Blank Transition Slide \\
% one slide transition, given all these facts, you need a good system to store
%computational sense, need store, upload, retrieval individual files that are big
%NEED A DATABASE
Databse Defn Slide \\
Database Principles Slide \\
% discuss this in relation of what I want to do
Normal Database Slide \\
%discuss NCBI databases & normal databases na RRL
P2P Database Slide \\
%discuss P2P databases
%and examples from RRL

Distributed Database Slide \\
%discuss distributed databases
%and examples from RRL like Seqtorr

Slide summarizing all the databases and technologies (yung old table dati) \\
% have a unified way of discussing each database
% what aspects of the database to discuss each
% extra features will come after
    Problems: % have one for each database system

Slide summarising all the problems found \\

Problem Slide:\\

    % sa unahan banda
    % pwede pang highlight na the reason you have all those databases, is because each has special needs
    % statement of the problem suggestion: merong conventional databases, like ncbi. May problems na inherent with those databases, na distributed databases are trying to solve. Until now, wala pang distributed database na
    
    % biotorrents, bittorious, wala pa na staple na ginagamit ng community. Bakit nagfafail yung distributed databases ngayon. Build a distributed database na stable enough and robust enough to be adapted by the sci. community, if di ginagamit usually dun nagkakaproblema
    
    % problem is two fold : one is to speed up upload and download (pagdistribute ng data centers)
    % two is to resolve problem of stability & usability of distributed databases (may gagamit), know lang the point of failure, from there come up with solution & metric to quantify their failures/successes
    % walang current implementation of distributed databases for genomics data, kasi di ginagamit (pero why) 
    
    
%PROBLEMS OF CLASSICAL DATABASES (ALL FOUR)
\item Genome researchers download entire database of genomic data
\item Genome researchers only need a portion of the database, but end up downloading everything 
% or having a hard time getting a portion of the database
% easier to download parts vs whole, whole vs parts

% not easy to navigate the database? 
% what is the big problem? too big data to download all at once?
% why is it a problem? speed of download? or size?

%PROBLEM OF DIST. DATABASES
    
Objectives\\
per problem statement connect to an objective \\

% problem x : objective x
% para mas clear yung objectives

Theoretical Framework \\
explain all network shit and tech aspect\\

%note we have one coherent flow for everything


slide for data model (get from seqtorr) \\

slide for data permissions \\

Bittorent Terms -> use a photo sana \\
\end{document}

Methodology \\
Have gantt chart din \\